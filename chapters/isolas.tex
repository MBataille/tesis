\chapter{Isolas of localized structures and Raman–Kerr frequency combs in micro-structured resonators (Chaos, Solitons \& Fractals 174, 113808)}

In section~\ref{sec:fra_LS} we introduced the concept of dissipative 
localized structures (LSs) which can be observed in a huge variety
of systems including fluid mechanics, vegetation patterns, biology and 
nonlinear optics [refs]. In the case of nonlinear optical systems,
a particularly interesting type of localized structure can be found,
the so-called optical solitons. Loosely speaking, a soliton
is a localized wave that emerges due to the balance between the nonlinearity
and dispersion, as well as the gain and loss of the system. Moreover,
solitons can propagate while preserving their shape and (usually) amplitude.
Furthermore, it has recently been discovered that the formation of solitons
in both fiber and micro-resonators gives rise to frequency combs, which are equally
spaced spectral lines. These frequency combs are remarkable
for technological purposes such as chip-scale atomic clocks \cite{Jost2015clock}, terabit
per second communication \cite{marin2017microresonator} and even the calibration of spectrometers
for exoplanet search \cite{suh2019searching}. From a theoretical perspective, the dynamics
of solitons in these optical resonator systems can be accurately described 
by the paradigmatic Lugiato-Lefever equation (LLE) \cite{lugiatolefever1987}.

In this chapter, we will study the formation of such structures in the
case of short solitons where the influence of the stimulated Raman scattering (SRS)
cannot be neglected and higher-order dispersion terms appear. In this case, a reduced model
in the form of a non-local Swift-Hohenberg equation is proposed to provide an
in-depth description and some analytical results. Employing numerical simulations and
continuation of the reduced model, we show that the SRS induces a forced symmetry
breaking leading to the motion of the LSs, as well as a disconnection between the different
branches of LSs. Consequently, the traditional snaking bifurcation scenario breaks
apart and instead, a family of isolas appears. Moreover, a numerical analysis of the
original model, the generalized LLE, confirms the formation of isolas of moving LSs due
to the reflection symmetry breaking, in agreement with previous studies \cite{burke2009swift,parra2014third}.
