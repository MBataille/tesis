\chapter{Isolas of localized structures and Raman–Kerr frequency combs in micro-structured resonators (Chaos, Solitons \& Fractals 174, 113808)}

In section~\ref{sec:fra_LS} we introduced the concept of dissipative 
localized structures (LSs) which can be observed in a huge variety
of systems including fluid mechanics, vegetation patterns, biology and 
nonlinear optics [refs]. In the case of nonlinear optical systems,
a particularly interesting type of localized structure can be found,
the so-called optical solitons. Loosely speaking, a soliton
is a robust localized wave that emerges due to the balance between the nonlinearity
and dispersion, as well as the gain and loss of the system. Moreover,
solitons can propagate while preserving their shape and (usually) amplitude.
Therefore, they provide a promising candidate for ultra-low-less optical communication
applications.

Furthermore, it has recently been discovered that the formation of solitons
in both fiber and micro-resonators gives rise to frequency combs, which are equally
spaced spectral lines. These frequency combs are especially useful
for technological purposes such as chip-scale atomic clocks, ultrafast 
photonic to analog conversion and even the calibration of spectrometers
for exoplanet search [refs]. From a theoretical perspective, the dynamics
of solitons in these optical resonator systems can be described 
by the paradigmatic Lugiato-Lefever equation (LLE) \cite{lugiatolefever1987}
with extreme accuracy [ref].

In this chapter, we will study the formation of such structures in the
case of short solitons where the influence of the stimulated Raman scattering (SRS)
cannot be neglected and higher-order dispersion terms appear. By means of
numerical simulations and continuation, we will show
that the SRS induces a forced symmetry breaking leading to the motion
of the LSs, as well as a disconnection between the different branches of LSs
which in turn form a family of isolas. 