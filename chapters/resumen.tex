\begin{resumen}
    Esta tesis está dedicada al estudio y caracterización de la dinámica de estructuras
    localizadas en sistemas no locales. Más específicamente, este trabajo abarca el estudio de dos
    tipos de estructuras localizadas en sistemas distintos, a saber, solitones disipativos
    en una cavidad de fibra de cristal fotónico, y quimeras espirales en arreglos bidimensionales
    de osciladores de fase acoplados no localmente. En ambos sistemas, se observa la aparición
    de estructuras localizadas móviles mediante distintos mecanismos, los cuales son correspondientemente identificados y discutidos.

    En el Capítulo~\ref{ch:intro}, se presenta una breve introducción a las estructuras localizadas y a los sistemas no locales,
    junto con el alcance y los objetivos de esta tesis. El Capítulo~\ref{ch:preliminary} proporciona los conceptos teóricos esenciales
    necesarios para comprender los capítulos siguientes, mientras que el Capítulo~\ref{ch:continuation} introduce el método de continuación
    numérica empleado en los capítulos posteriores.

    La primera parte de la investigación realizada en esta tesis se compone de los Capítulos~\ref{ch:isolas} y~\ref{ch:filtering},
    y está dedicada al estudio de solitones disipativos en cavidades de fibra de cristal fotónico bajo distintos efectos no locales.
    En particular, el Capítulo~\ref{ch:isolas} está dedicado a la caracterización y análisis de solitones brillantes y oscuros
    móviles sujetos al efecto Raman y considerando dispersión de cuarto orden. Un sistema similar
    es considerado en el Capítulo~\ref{ch:filtering}, con la diferencia de que los dos términos anteriores
    son reemplazados por un filtro espectral, que surge de la adición de una rejilla de fibra al sistema.

    La segunda parte está compuesta por los Capítulos~\ref{ch:spirals} y~\ref{ch:travelingspirals},
    y se centra en quimeras espirales móviles en poblaciones de osciladores de fase acoplados. Comenzando
    con el Capítulo~\ref{ch:spirals}, se revela la existencia de estas estructuras, y se caracteriza su
    dinámica basada en simulaciones numéricas. Posteriormente, el Capítulo~\ref{ch:travelingspirals}
    profundiza el análisis a través del método de continuación numérica y métodos semi-analíticos, proporcionando
    una descripción detallada de la aparición y estabilización de quimeras espirales móviles.

    Finalmente, el Capítulo~\ref{ch:conclu} presenta las principales conclusiones de esta tesis.
\end{resumen}

\begin{abstract}
    This dissertation is devoted to the study and characterization of the dynamics of localized structures
    in non-local systems. More specifically, we focus on two different kind of localized structures
    arising in different systems, 
    namely, dissipative solitons in a photonic crystal fiber cavity, and
    spiral chimeras in two-dimensional arrays of nonlocally coupled phase oscillators. In both 
    systems, we observe the emergence of moving localized structures due to distinct mechanisms,
    which we identify and discuss.

    In Chapter~\ref{ch:intro}, a brief introduction to localized structures and nonlocal systems
    is presented, along with the scope and objectives of this thesis. Chapter~\ref{ch:preliminary}
    provides the essential theoretical concepts needed to understand the following chapters, 
    while Chapter~\ref{ch:continuation}
    introduces the numerical continuation method employed in the subsequent Chapters.
    
    The first part of the research made in this dissertation consists of Chapters~\ref{ch:isolas} and~\ref{ch:filtering},
    and is dedicated to the study of dissipative solitons in photonic crystal fiber cavities under different nonlocal effects.
    In particular, Chapter~\ref{ch:isolas} is dedicated to the characterization and analysis of moving bright and dark
    subject to the Raman effect and considers fourth order dispersion. A similar system 
    is considered in Chapter~\ref{ch:filtering}, with the difference that the two former terms 
    are replaced by a spectral filter, arising from the addition of a fiber grating to the system.
    
    The second part is composed of Chapters~\ref{ch:spirals} and~\ref{ch:travelingspirals}, 
    and focuses on moving spiral chimeras in populations of coupled phase oscillators. Starting
    with Chapter~\ref{ch:spirals}, the existence of these structures is revealed, and their 
    dynamics are characterized based on numerical simulations. Subsequently, Chapter~\ref{ch:travelingspirals}
    deepens the analysis through numerical continuation and semi-analytical methods, providing
    a detailed description of the emergence and stabilization of moving spiral chimeras.
    
    Finally, Chapter~\ref{ch:conclu} presents the main conclusions of this dissertation.
\end{abstract}