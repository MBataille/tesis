\chapter{Moving spiral wave chimeras (Physical Review E, 104(2), L022203)}

So far, we have devoted our investigation to drifting one-dimensional localized states in a non-local optical system. 
In those two cases, the motion itself was to be expected since a reflection
symmetry-breaking term was present. Nevertheless, such a term is not always
necessary for there to be moving localized states. Indeed, in this chapter, 
we shall show that the motion can arise spontaneously
in an isotropic system of coupled phase oscillators.

In section~\ref{sec:phase_oscillators}, we emphasized the importance
and universality of phase oscillators, as they offer a comprehensive
description of various self-sustained oscillators while disregarding amplitude 
dynamics. Such oscillators can be found all around us,
in flashing fireflies, circadian cycles, lasers, electrical equipment, and even within us
like neurons in the brain and pacemaker cells in the heart. Although the dynamics
of an individual oscillator is already a challenging problem, as evidenced
by the Nobel Prize awarded to Hodgkin and Huxley for their neuron model
\cite{hodgkin1952quantitative},
exploring the dynamics of a population of coupled phase oscillators
introduces greater complexity and richness. One fundamental question that
arises in these interacting oscillators systems is whether they will synchronize, 
and if so, to what degree?

While numerous authors made tremendous advances on this topic, one of the
most notable is probably Kuramoto who provided a simple yet powerful model 
for describing coupled oscillators several decades ago \cite{kuramoto1975model}. Even today, the Kuramoto
model and its modifications remain a very active research field [citas].
Remarkably, when Kuramoto and Battogtokh extended the model considering non-local coupling instead, he observed an incongruous state
of partial synchrony where coherent (synchronized) and incoherent (desynchronized) domains coexisted \cite{kuramoto2002coexistence}.
One would expect that coupling identical oscillators would only yield a completely coherent
state, yet they showed that the symmetry could be spontaneously broken and the system
would self-organize into two different and coexisting domains, what we now call a chimera state \cite{abrams2004chimera}.

Even more surprisingly, two-dimensional simulations of the Kuramoto-Battogtokh model
revealed a planar chimera state in the form of a spiral wave. This spiral wave chimera
presented an incoherent core at the tip of the spiral where oscillators were desynchronized
while the remaining oscillators located in the spiral arms were synchronized. Similar
patterns were already observed in reaction-diffusion systems except for one significant
difference, the phase singularity located in the tip of the spiral was now replaced by
an incoherent core. Since then they have been repeatedly observed in a variety of systems
both numerically [citas] and experimentally [citas]. 


2d extension: spiral wave chimera. similat to spiral wave, experiments, relevance

in this chapter
importance of synchronization. 
synchronization: hugyens (history), kuramoto (predictions, see section xx).
one could expect they synchronize but chimera state! kuramoto battogtokh,
experiments, etc. introduce spiral wave chimera

in this chapter...

Coupled oscillators, synchronization. Very important in many contexts, brain, heart,
power grid, etc.

Unexpectedly, a different state was also found: the chimera state. explain
They have been predicted and experimentally observed in a plethora of systems
[refs]. 

In the weak coupling limit, these populations of nonlinear oscillators can be accurately
described by a simple yet powerful model: the Kuramoto model. 


This second half will be devoted to the study of coupled phase oscillators. As
stated previously, in section~\ref{sec:phase_oscillators}, the dynamics of
a non-linear oscillator, in the limit of weak coupling, can be reduced
to a single cyclic variable: the phase. Therefore, an adequate and general model
for a population of coupled oscillators  

[synchronization in the brain] \cite{erra2017neuralsynchronization}

\includepdf[pages={-}]{chapters/movingspirals.pdf}
