\chapter{Introduction}

\label{ch:intro}

Almost a century ago, our understanding of elementary or quantum particles went
through a complete change of paradigm. Indeed, what was originally thought
to be a microscopic speck of matter with a well-defined position, momentum and size
was discovered instead to be a localized field that occupies all space, 
and whose amplitude is related to the probability 
density~\cite{merzbacher1998quantum,messiah2014quantum,gross1962particle}.
As a consequence,
the former quantities of position and momentum were replaced by expectation values.
Conversely, the concept of a particle as a localized field was extended to 
meso- and macroscopical systems, typically described by extended fields.
In fact, Nicolis and Prigogine argued that the energy transfer present in these systems
allows for a rich variety of complex and stable structures to form, including localized 
structures~\cite{prigogine1977self, pismen2006patterns}.
These localized states or particle-like solutions are thus supported by a robust balance between 
gains and losses and have been observed in a variety of systems due to their 
universality~\cite{umbanhowar1996localized,aranson2006patterns,minardi2010three,verschueren2013spatiotemporal,yi2018imaging}. 

Traditionally, the mathematical description of localized structures has been carried out using
reaction diffusion models or their generalizations with higher order 
derivatives~\cite{tlidi1994localized,coullet2000stable,knobloch2015spatial}.
As the name suggests, the spatial coupling occurs through diffusive (or hyperdiffusive) terms and, 
therefore, is solely local. Nevertheless, various optical~\cite{gelens2007impact}, 
neural~\cite{coombes2005waves} and even vegetation~\cite{lefever1997origin} systems present a more 
complex and far-reaching coupling, usually called nonlocal coupling. In these cases,
it has been found that the nonlocal term gives rise to richer dynamics and is often responsible for the
stabilization of particle-like solutions~\cite{fernandez2013strong, clerc2020time}. Mathematically, the nonlocal term is usually modeled as an
integral operator, which strongly limits the analytical and numerical tools available to study these systems.
Consequently, the dynamics of localized structures in nonlocal systems are still not well understood.

The aim of this dissertation is to provide an in-depth study and characterization of
the dynamics of localized structures in non-local systems. In particular, we will focus 
on two different systems: dissipative solitons in a photonic
crystal fiber cavity, and spiral chimeras in two-dimensional networks of heterogeneous phase oscillators. 
In both cases, we observe the emergence of uniformly moving localized structures due to different mechanisms,
which we will identify and analyze.

\section{Contents}

This dissertation is a compilation of four articles that have been published or submitted to peer-reviewed journals.
For each article, there is an associated chapter containing a brief introduction to the subject, 
along with a short discussion at the end. 

The first three chapters of this dissertation are dedicated
to the theoretical and numerical background necessary for the subsequent chapters. More specifically,
Chapter~\ref{ch:intro} corresponds to the present introduction, and provides an overview of the context and 
scope of the dissertation. Chapter~\ref{ch:preliminary} introduces the main concepts and theory needed 
to approach the current study. In Chapter~\ref{ch:continuation}, an introduction to numerical continuation,
the main numerical method used in the following articles, is presented. 

The following four chapters contain the core of the dissertation, which is divided into two main parts. The first
part contains Chapters~\ref{ch:isolas} and~\ref{ch:filtering}, and focuses on dissipative solitons in photonic crystal 
fiber cavities, under different asymmetric nonlocal effects. Starting with Chapter~\ref{ch:isolas},
the formation of isolas of solitons due to the Raman effect is studied. In Chapter~\ref{ch:filtering}, a similar 
system is considered under a different nonlocal coupling, arising from the addition of a spectral filter.

On the other hand, the second part of the dissertation is devoted to the study of moving spiral wave chimeras
in a two-dimensional array of nonlocally coupled phase oscillators. In particular, Chapter~\ref{ch:spirals}
reveals the existence of these structures and characterizes their dynamics based on direct numerical simulations.
Chapter~\ref{ch:travelingspirals} extends the analysis through numerical continuation and semi-analytical methods,
and provides a detailed description of the bifurcations that lead to the creation, stabilization and disappearence of
moving spiral chimeras. Finally, Chapter~\ref{ch:conclu} presents the main conclusions of this thesis.


\section{Objectives}
The main objective of this dissertation is to study and characterize the creation, stabilization and disappearence
of propogative localized structures in nonlocal systems. To achieve this, the following specific objectives were proposed.
\begin{itemize}
  \item Identify and understand the different physical mechanisms responsible for the propagation of
  localized structures.
  \item Classify the different types of dynamics through an extensive exploration of the parameter space.
  \item Develop and implement a numerical method for the continuation of complex traveling solutions in
  one- and two-dimensional nonlocal systems.
  \item Identify and characterize the bifurcations that lead to the creation, stabilization and disappearence 
  of localized structures through the use of numerical continuation techniques.
\end{itemize}

\section{Common abbreviations}

\begin{itemize}
  \item {\bf LS}: Localized structures
  \item {\bf DS}: Dissipative soliton
  \item {\bf SHE}: Swift-Hohenberg Equation
  \item {\bf LLE}: Lugiato-Lefever Equation
  \item {\bf NLS}: Nonlinear Schrödinger Equation
  \item {\bf KdV}: Korteweg-de Vries Equation
  \item {\bf HSS}: Homogeneous Steady State
  \item {\bf SRS}: Stimulated Raman Scattering
\end{itemize}

\section{Contribution statement}

\subsection{Isolas of localized structures and Raman–Kerr frequency combs in micro-structured resonators (Chaos, Solitons \& Fractals 174, 113808)}
Marcel G. Clerc and Mustapha Tlidi conceptualized the study and designed the research. 
{\bf Martin Bataille-Gonzalez} implemented and performed numerical simulations and continuation. 
Marcel G. Clerc and Mustapha Tlidi wrote the manuscript. {\bf Martin Bataille-Gonazalez} 
contributed to editing and revising the manuscript.


\subsection{Dissipative Soliton Combs with Spectral Filtering. (submitted to Physical Review A)}
Marcel G. Clerc and Mustapha Tlidi conceptualized the study and designed the research. {\bf Martin Bataille-Gonzalez},
Bilal Kostet and Youri Soupart performed research (numerical integration, implementation of numerical continuation, and analysis of numerical data).
Bilal Kostet and Mustapha Tlidi developped the analytical framework.
Bilal Kostet, Youri Soupart, Mustapha Tlidi and Marcel Clerc wrote the manuscript. 
{\bf Martin Bataille-Gonazalez} contributed to editing and revising the manuscript.

\subsection{Moving spiral wave chimeras (Physical Review E, 104(2), L022203)}
Oleh Omel'chenko conceptualized the study and designed the research. {\bf Martin Bataille-Gonzalez} performed research (numerical integration, and analysis of numerical data).
Marcel G. Clerc supervised the research and provided critical feedback. Oleh Omel'chenko and Marcel G. Clerc wrote the manuscript. 
{\bf Martin Bataille-Gonazalez} contributed to editing and revising the manuscript.

\subsection{Traveling spiral wave chimeras in coupled
oscillator systems: emergence, dynamics, and
transitions (New Journal of Physics 25, 103023)}
Oleh Omel'chenko conceptualized the study and designed the research. 
{\bf Martin Bataille-Gonzalez} and Oleh Omel'chenko implemented and performed numerical simulations and continuation.
Marcel G. Clerc and Edgar Knobloch supervised the research and provided critical feedback. Oleh Omel'chenko and Edgar Knobloch wrote the manuscript. 
{\bf Martin Bataille-Gonazalez} and Marcel G. Clerc contributed to editing and revising the manuscript.