\chapter{Introduction}
Las partículas o corpúsculos han sido un concepto fundamental y transversal en la física. Clásicamente se describen como un punto material con una masa y posición bien definida. Sin embargo, con el desarrollo de la mecánica cuántica, hemos aprendido que las partículas microscópicas son soluciones localizadas de un campo corrrespondiente a la amplitud de la probabilidad. Por otro lado, en sistemas macroscópicos fuera del equilibrio, producto del balance entre la inyección y disipación de energía, estos sistemas pueden exhibir soluciones localizadas que, en analogía con el caso anterior, usualmente se denonominan soluciones tipo partícula \cite{LS:b}. Estas estructuras localizadas han sido observadas en diversos sistemas de dinámica de fluidos, óptica, química e incluso ecología \cite{LS:a, LS:b}. Dependiendo del contexto físico en que se observan también reciben el nombre de solitones disipativos, patrones localizados, quimeras, entre otros. 

Tradicionalmente, la descripción matemática de las estructuras localizadas se ha realizado usando modelos de reacción difusión \cite{RD:VanagEpstein, RD:Winfree, RD:Barkley}. Como el nombre sugiere el acoplamiento espacial ocurre mediante un término de difusión y, por lo tanto, es puramente local. No obstante, diversos sistemas ópticos, neuronales e incluso en vegetación presentan un acoplamiento más complejo y de largo alcance, usualmente llamado acoplamiento no local \cite{NonLocal:Optics, NonLocal:Neuron, NonLocal:Vegetation}. En estos casos se ha encontrado que el término no local es responsable de la estabilización de las soluciones tipo partícula \cite{NonLocal:Stabilize, Raman:Stabilize} y por tanto es fundamental en el entendimiento de estas soluciones.


{\bf En esta tesis nos enfocaremos en la dinámica de las estructuras localizadas en sistemas no locales}. En particular, buscaremos entender los mecanismos que permiten la propagación de estas soluciones tipo partícula. Para lograrlo, estudiaremos dos sistemas diferentes: solitones disipativos en una cavidad de cristal de fibra fotónico \cite{Raman:Agrawal}, y quimeras espirales en redes bidimensionales de osciladores de fase heterogéneos. 

En el primer caso, estudiaremos solitones brillantes y oscuros en resonadores de cavidad de cristal de fibra óptica. Los solitones disipativos temporales han recibido una enorme atención estas últimas décadas por su capacidad de generar {\em frequency combs} o peines de frecuencia que han revolucionado diversas áreas de la ciencia y tecnología, en particular la espectroscopía de alta precisión y metrología \cite{Raman:FB,Raman:PH}. La mayor parte de estos esfuerzos científicos se han centrado en la generación de solitones mediante un balance entre la nolinearidad Kerr del material y el acoplamiento local temporal por ejemplo debido a la dispersión \cite{Raman:AkhAnk}. No obstante, en materiales amorfos como lo son los cristales de fibra óptica emerge un acoplamiento no local temporal debido a una respuesta retardada del material a la excitación electromagnética que se conoce como scattering estimulado de Raman \cite{Raman:SKV, Raman:LCC}.  Gracias al efecto Raman es posible la estabilización de estas estructuras localizadas \cite{Raman:Stabilize}. En este trabajo, buscaremos estudiar cómo son afectadas las estructuras localizadas debido al efecto Raman y la nolinearidad Kerr, y en particular caracterizar precisamente cómo se auto-organizan estas soluciones en función de los parámetros del sistema.

Recientemente hemos podido reducir el modelo paradigmático de Lugiato-Lefever en torno a la emergencia de la biestabilidad encontrando así la ecuación de Swift-Hohenberg. De forma preliminar, hemos encontrado solitones brillantes y oscuros en la ecuación de Swift-Hohenberg con efecto Raman, en la región de coexistencia del estado homogéneo y el estado patrón. Además, debido al acoplamiento no local por efecto Raman, la simetría del sistema se rompe, lo que nos permite encontrar una propagación de las estructuras localizadas y una desconexión de las ramas de solución dando origen a una cadena de isolas, ver figura.


En el segundo caso, analizaremos estados ligados de dos quimeras espirales en redes de osciladores acoplados espacialmente de forma no local. Las quimeras espirales se caracterizan por tener un núcleo incoherente donde los osciladores están desincronizados, rodeado por una estructura coherente en forma de espiral donde los osciladores están sincronizados. Estas soluciones fueron reportadas por primera vez por Kuramoto y Shima hace dos décadas al acoplar osciladores de forma no local \cite{SpiralChimera:Kuramoto}. Durante estas últimas dos décadas han sido ampliamiente estudiadas en diversos sistemas neuronales, eléctricos e incluso en ecología \cite{SC:FHN2018, SC:Eco2018, SC:BA2020}, además de ser observadas experimentalmente en redes de osciladores químicos \cite{SC:Totz2017, SC:Totz2020}. Sin embargo, la mayor parte de estos estudios se han centrado en quimeras estacionarias y poco se conoce sobre las quimeras propagativas \cite{SC:Omelchenko2020, SC:Bataille}. 
 En trabajos preliminares, encontramos un nuevo tipo de quimera espiral: un estado ligado de dos espirales propagativas que pueden moverse en una línea recta o siguiendo una trayectoria más compleja.

Mediante simulaciones preliminares hemos logrado encontrar 3 clases de quimeras espirales propagativas: simétricas, asimétricas y cicloidales, junto con su región de estabilidad correspondiente, ver figura~\ref{fig:SpiralChimera}. Las quimeras simétricas poseen simetría de reflexión y se propagan en la dirección de su eje de simetría mientras que en las quimeras asimétricas, una espiral se vuelve más grande que la otra y se propagan en una dirección inclinada. Por último las espirales cicloidales presentan, además de un movimiento de traslación, una oscilación de sus núcleos que también suele llamarse {\em meandering} \cite{SC:Bataille}.

\section{Contents}

\section{Common abbreviations}

LS. Localized structures
DS. Dissipative soliton
SHE. Swift-Hohenberg Equation
LLE. Lugiato-Lefever Equation
\section{Contribution statement}