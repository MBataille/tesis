\chapter{Preliminary concepts}

\section{Dynamical Systems}

[introducir bien el tema]. 

The general form for a dynamical system will be the following.

\begin{equation}
    \dfrac{d\bm{u}}{dt} = \bm{f}(\bm{u}, \eta), \qquad \bm{f} : \mathbb{R}^N \times \mathbb{R}^{n_\eta} \to \mathbb{R}^N
    \label{eq:def_ds}
\end{equation}

Here, $\bm{u}$ represents the state vector of the system, it might correspond to the
concentrations of different chemicals, the population of certain species or the amplitude
of an electric field. The temporal evolution of the state of system is thus determined by 
the vector function $\bm{f}$. This function may in turn depend on one or several control
parameters $\eta$ relevant to the modeled experiment such as
[Agregar parametros]

In this thesis, we will consider different dynamical systems represented by Eq.~(\ref{eq:def_ds}) where the function
$\bm{f}$ will be {\em nonlinear}. Although one might argue that, at a fundamental level, the physical laws
that describe the evolution of a system (Schrödinger equation for example) are linear, when one looks at meso- or macroscopical
systems, nonlinear terms naturally arise due to the coarse-graining of the microscopical degrees of freedom [ref Karadr].


In the case of a nonlinear dynamical system, it becomes extremely difficult, and often impossible, to find
general explicit solutions of Eq.~(\ref{eq:def_ds}). But it turns out that in most cases an in depth
description of the model can be provided by studying only the steady states ($\bm{f}(\bm{u}, \eta) = 0$) and their qualitative changes as parameters 
are varied. In other words, the problem can be reduced as to find the {\em equilibria} and {\em bifurcations} of the system.
In the following section, we shall describe the simplest bifurcations a system can experience.

\section{Bifurcations}

Qualitative changes of equilibria as parameters are varied.

\subsection{Saddle-Node bifurcation}

Saddle-node or fold bifurcations provide the simplest mechanism for which a pair of stable and
unstable equilibria can be created (or destroyed) as the control parameter is changed. They arise
in a wide variety of systems [insert ref]. The corresponding normal form is the following.

\begin{equation}
    \dfrac{du}{dt} = \eta - u^2
\end{equation}

For $\eta > 0$, the system presents two equilibria $u_{\pm} = \pm\sqrt{\eta}$, where $u_+$ is
stable and $u_-$ unstable. An interesting case occurs when $\eta = 0$, at which point $u = 0$ is
half-stable (stable for positive perturbations and unstable for negative perturbations). Lastly,
for $\eta < 0$ there are no equilibria. In short, as the bifurcation parameter $\eta$ is decreased (increased)
starting from positive (negative) values, the two equilibria attract (repel) each other and suddenly annihilate (appear).

[insert example]

\subsection{Pitchfork bifurcation}

[insert example]

\subsection{Hopf bifurcation}

[insert example]

\section{Localized Structures}
\label{sec:fra_LS}

\section{Chimera states}
\subsubsection{Phase oscillators}

\subsection{Kuramoto model}

\subsection{Spiral wave chimeras}