\chapter{Dissipative Soliton Combs with Spectral Filtering. (IEEE Journal of Quantum Electronics, XX(X), X-XX)}

The previous chapter focused on the formation of dissipative solitons in optical
resonators when a non-local term due to Raman scattering is incorporated into the Lugiato-Lefever equation. The
integral form of this additional term poses several analytical and numerical difficulties.
For instance, in the case of the numerical continuation, where a very
large system of equations needs to be solved, the addition of the integral term transforms the extremely sparse
system into a dense system, increasing the computational cost in terms of both memory and time
by orders of magnitude.

An alternative approach to deal with the integral form of a non-local operator is to represent
it by an infinite Taylor series, also referred to as a gradient expansion. Moreover, 
assuming that the kernel function is sufficiently localized, the expansion can be
safely truncated and only the first few relevant terms can be kept. This way, the integral operator can be
approximated by a few differential operators with coefficients that depend on
the kernel. Consequently, the representation in terms of differential operators
allows for a much easier analytical investigation and numerical characterization of the system.
For this reason, this gradient approximation to a nonlocal operator has been widely 
preferred in studies of vegetation patterns \cite{lefever1997origin}, [add more].

In what follows, we apply the alternative approach stated above to investigate
the formation of solitons and patterns in a fiber ring resonator with a spectral filter.
To justify the use of a truncated gradient expansion, the case of a large-bandwidth filter
(or equivalently, a localized temporal filter) is considered. Starting from an infinite-dimensional
Ikeda map, a partial differential equation is derived in the form of a Lugiato-Lefever equation with additional
first and second derivative terms from the gradient approximation. Similarly, as in the previous chapter,
a forced reflection symmetry breaking takes place, leading to the drift of both localized and periodic states.
Furthermore, it is observed that, in the presence of the filter, the stability region of LSs is significantly reduced
and, the Hopf bifurcation point that gives rise to breathing solitons is shifted towards a larger pumping.
Lastly, through numerical continuation, the persistence of the snaking bifurcation scenario in
the presence of the diffusive term is revealed along with its destruction upon the addition of the reflection symmetry-breaking term.

\includepdf[pages={-}]{chapters/filtering.pdf}

\section{Perspectives}

In this article, we showed that, under the addition of a spectral filter, a series of
derivative terms must be incorporated into the LLE. Then, assuming a sufficiently
localized filter, we justified a second-order truncation of the series and studied 
the effects of the additional terms with coefficients $\alpha_1, \alpha_2$ on the stability and dynamics of both localized 
and periodic structures. However, a more extensive exploration of the system's parameters 
remains to be performed. In this direction, a phase diagram showing the stability boundaries
in terms of these two coefficients would provide a more complete description and be
especially helpful for experimentalists. On this note, it would also be interesting to
compute the coefficients corresponding to previous experimental studies \cite{bessin2019gain}
and test whether our approximation is still valid in a real experiment. If this is not the case,
then a detailed numerical analysis via continuation of the integrodifferential 
equation deserves to be performed and compared with the results shown in this chapter.
