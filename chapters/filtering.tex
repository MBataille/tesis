\chapter{Dissipative Soliton Combs with Spectral Filtering. (IEEE Journal of Quantum Electronics, XX(X), X-XX)}

In the previous chapter, we studied the formation of dissipative solitons in optical
resonators when a non-local
term due to Raman scattering is incorporated into the Lugiato-Lefever equation. The
integral form of this additional term poses several analytical and numerical difficulties.
For instance, in the case of the numerical continuation, where one needs to solve a very
large system of equations, the addition of the integral term transforms the extremely sparse
system into a dense system, increasing the computational cost in terms of both memory and time
by several orders of magnitude.

An alternative approach to deal with the integral form of a non-local operator is to represent
it by an infinite Taylor series, also referred to as a gradient expansion. Moreover, 
assuming that the kernel function is sufficiently localized, one can safely truncate the expansion
and keep only the first few relevant terms. This way, the integral operator can be
effectively replaced by a few differential operators with coefficients that depend on
the kernel function.

In this chapter, we will follow the alternative approach stated above to investigate
the formation of soliton and patterns in a fiber ring resonator with a spectral filter. 
We will consider the case of a large-bandwidth filter (or equivalently, localized temporal filter)
which allows for a truncated gradient expansion.