\chapter{Traveling spiral wave chimeras in coupled
oscillator systems: emergence, dynamics, and
transitions (New Journal of Physics 25, 103023)}

\label{ch:travelingspirals}

Previously, we demonstrated that even with a symmetric coupling kernel, two-core spiral chimeras develop different
types of sustained motion. However, the fundamental question of their emergence remained
unresolved due to two main challenges. Firstly, the analytical results for the model proved
extremely difficult to obtain, necessitating the use of numerical methods, which leads to the second 
issue. The computational cost, in both time
and memory, of performing numerical continuation in a two-dimensional non-local model is exceedingly high.

This chapter addresses the previously stated question regarding the emergence of two-core spirals by analyzing two distinct coupling kernels: a
top-hat and a sinusoidal kernel. In the case of the top-hat coupling, a complete bifurcation diagram of symmetric two-core spirals
is provided via numerical continuation. Conversely, the sinusoidal kernel
permits a semi-analytical approach \cite{omel2018stability,xie2015twisted,omel2019chimerapedia}, which allows for a detailed 
characterization of the stability of the solutions and their bifurcations. Notably,
we expose here that the emergence of spirals is explained by a series of bifurcations from a homogeneous
state. Furthermore, it is observed that the transition from static to moving spiral chimeras
triggers the formation of filaments within their incoherent core, through a sequence of saddle-node
bifurcations, similar to the slanted snaking bifurcation observed in other non-local systems
\cite{firth2007proposed,firth20017homoclinic,barbay2008homoclinic,thiele2013localized}. 

\includepdf[pages={2-}]{chapters/travelingspirals.pdf}

\section{Perspectives}

We have investigated the emergence of moving two-core spiral chimeras for a top-hat
and a trigonometric kernel. In particular, for the former kernel, it has been shown
that new filaments develop in the incoherent core after a pair of folds in the bifurcation diagram.
On the other hand, for the trigonometric kernel, a more in-depth analysis was carried out
revealing all equilibria for $\alpha=0$ and their transitions. In particular, the sequence
of three symmetry breaking bifurcations originating from the partially coherent uniform state, 
explaining the emergence of two-core spirals, has been uncovered. Through an extension of the
semi-analytical method, the case of symmetric spirals for $\alpha \neq 0$ was studied, unveiling a similar
bifurcation diagram as for the top-hat coupling, confirming the robustness of our results.
Therefore, it can be expected that the results shown here may, to a certain extent, persist
for different and more realistic coupling kernels. 

The totality of the results presented here are based on a detailed analysis of the equilibria
and their transitions. Particularly, the different possible configurations of two-core spirals
both symmetric and asymmetric were found for a given set of parameters. Nevertheless,
the time evolution from a certain initial condition to the final stable configuration remains
unknown. A systematic study of the previously described transient dynamics, could shed light on how they spiral
chimeras interact between each other and form stable two-core (even four-core and beyond)
states. Therefore, this research direction could provide a complementary understanding on
the formation of multi-core spiral chimeras, and deserves to be pursued.