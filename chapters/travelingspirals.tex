\chapter{Traveling spiral wave chimeras in coupled
oscillator systems: emergence, dynamics, and
transitions (New Journal of Physics 25, 103023)}

Previously, we demonstrated that even with a symmetric coupling kernel, two-core spiral chimeras develop different
types of sustained motion. However, the fundamental question of their emergence remained
unresolved due to two main challenges. Firstly, analytical results for the model proved
extremely difficult to obtain, necessitating the use of numerical methods, which leads to the second 
issue. The computational cost, in both time
and memory, of performing numerical continuation in a two-dimensional non-local model is exceedingly high.

This chapter addresses the former question by analyzing two distinct coupling kernels: a
top-hat and a sinusoidal kernel. In the case of the top-hat coupling, a complete bifurcation diagram of symmetric two-core spirals
is provided via numerical continuation. Conversely, the sinusoidal kernel
permits a semi a semi-analytical approach, which allows for a detailed 
characterization of the stability of the solutions and their bifurcations. Notably,
the emergence of spirals is shown to be explained by a series of bifurcations from a homogeneous
state. Furthermore, it is observed that the transition from static to moving spiral chimeras
triggers the formation of filaments within their incoherent core, through a sequence of saddle-node
bifurcations remiscent of the homoclinic snaking scenario [cita]. 

\includepdf[pages={2-}]{chapters/travelingspirals.pdf}

\section{Perspectives}

[a more realistic coupling]
[interaction between spiral wave chimeras could explain the formation of a bound state of 
2core spirals and beyond.]