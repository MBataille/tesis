\chapter{Conclusions}
\label{ch:conclu}

This thesis was devoted to the study and characterization of the dynamics
of localized structures in different one- and two-dimensional nonlocal systems.
To achieve this, we explored the effect of both asymmetric and symmetric nonlocal
couplings in optical resonator and phase oscillator systems, respectively. In 
both cases, the emergence of traveling LSs was observed, and the main bifurcations
leading to their creation, stabilization and disappearence were numerically identified.

In the case of fiber ring resonators, we considered variations of the 
Lugiato-Lefever equation with different asymmetric coupling terms:
the Raman effect, and spectral filtering. The latter was modeled
as a gradient expansion of the nonlocal operator. In both cases, these
reflection symmetry-breaking terms were found to be responsible
for the drift of both localized and periodic states. Additionally,
numerical continuation revealed that these terms caused 
the destruction of the traditional homoclinic snaking 
bifurcation scenario, which led to the formation of isolas of LSs.
Nevertheless, not all of the bifurcations present in the diagram were completely
identified, and further work is needed in this direction.

Subsequently, we studied the dynamics of spiral wave chimeras in a
two-dimensional network of nonlocally coupled phase oscillators. 
Unexpectedly, we found that this system exhibits moving spiral chimeras,
in spite of the coupling function beeing symmetric and isotropic. 
The motion of these structures was, therefore, found to be driven by a spontaneous
symmetry breaking. Moreover, we identified three different types of moving
spirals chimeras according to their motion, and found their respective
stability region. The case of symmetric spirals was found to be the most
approachable in terms of computational cost, and thus was studied in more detail.
More specifically, we obtained the corresponding bifurcation diagram showcasing
its main bifurcations. In addition, we considered the effect of a sinusoidal coupling,
since it allowed to perform semi-analytic calculations. In this case, we found
the sequence of bifurcations which lead to the creation of moving spiral chimeras
from a homogeneous state.



