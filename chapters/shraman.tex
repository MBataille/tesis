\chapter{Moving Solitons in the Lugiato-Lefever equation.}

In section~\ref{sec:fra_LS} we introduced the concept of dissipative localized structures (LSs). Here, we will study the formation
of such structures in nonlinear optical systems where they are often called optical or cavity solitons. 
More specifically, we will analyze the paradigmatic Lugiato-Lefever equation (LLE) \cite{lugiatolefever1987} used to describe fiber resonators.

\section{Lugiato-Lefever equation.}

In 1987, Lugiato and Lefever proposed a simple yet extremely rich nonlinear partial differential equation to study the formation of patterns and localized states
in the framework of nonlinear optics \cite{lugiatolefever1987}. They considered a cavity filled with a nonlinear medium in the low transmission (or high quality) limit
driven by a continuous wave. In order to keep the equation as simple as possible, they considered a cubic nonlinearity which is characteristic of Kerr media. Moreover,
In virtue
of the low dissipation limit, they originally neglected the longitudinal variable $z$ (along which light propagates) and kept only the transversal plane $x-y$ as spatial
variables in the equation. In contrast, a longitudinal (or temporal) LLE was later formulated by Haelterman and his colleagues [ref], where only the longitudinal 
coordinate becomes relevant. The main difference between these two equations is that in the former, a transversal Laplacian appears due to diffraction of the light, whereas
in the latter, a longitudinal Laplacian appears due to dispersion of the light. However, from a mathematical point of view, they are the same equations.

In our case, we will consider the longitudinal LLE corresponding to Eq.~(). Moreover, we will consider the Raman effect... 



Following
these assumptions, they neglected the longitudinal $z$-coordinate and kept only the transversal $x,y$ dependence arising from the diffraction.  
they derived an equation for the complex envelope of the electric field $E(x, y, t)$ corresponding to Eq.~(\ref{lle:lle_original}),
where $E_{in}$ corresponds to the input field, $\theta$ the detuning (proportional to the distance between the frequency of the cavity and that of the driving field).

\begin{equation}
    \dfrac{\partial E}{\partial t} = E_{in} - (1 + i\theta) E + i |E|^2 E + i\nabla^2_{x,y} E
    \label{lle:lle_original}
\end{equation}

Note that in equation~(\ref{lle:lle_original}), there appears a transverse Laplacian $\nabla^2_{x,y}$

\section{Raman effect.}

\section{A reduced model.}

\section{Isolas and traveling solitons.}

\section{Oscillatory bound states.}
